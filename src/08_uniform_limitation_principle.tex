% !TeX root = ./document.tex
\documentclass[document]{subfiles}
\begin{document}

\chapter{Принцип равномерной ограниченности}

Принцип равномерной ограниченности, тут будет много теорем, они все связаны с первой, а всё вместе это второй кит линейного функционального анализа.

\begin{theorem}[принцип равномерной ограниченности]
    $(X, \norm{\cdot})$ --- банахово, $(Y, \norm{\cdot})$ --- нормированное, $\seq{U_\alpha}_{\alpha \in A}, U_\alpha \in \B(X,Y)$
    \[ \forall x \in X \: \sup_{\alpha \in A} \norm{U_\alpha x} < +\infty \Rightarrow \: \exists M > 0 : \norm{U_\alpha} \leq M \: \forall \alpha \in A \]
    
\end{theorem}

Причём тут равномерность? 
\begin{gather*}
    \norm{U_\alpha} = \sup_{\seq{x \in X : \norm{x} \leq 1}} \norm{U_\alpha x}  \\ 
    \sup_{\alpha \in A} \sup_{\seq{x \in X : \norm{x} \leq 1}} \norm{U_\alpha x} < + \infty
\end{gather*}
предполагаем для одного икса, а оказывается, что можно взять $\sup$ по единичной сфере, а потом ещё раз взять $\sup$, и это фантастически полезно и в то же время странно. Начнём с простой леммы.

\begin{lemma}
    $(X, \norm{\cdot}), (Y, \norm{\cdot})$ ---  нормированные
    \begin{gather*}
        U \in \Lin(X,Y), \: \exists \varepsilon > 0, R > 0, a \in X : \\
        U(B_\varepsilon(a)) \subset \overline{B_R(0)} \Rightarrow U \in \B(X,Y) \\
        \norm{U} \leq \frac{2R}{\varepsilon}
    \end{gather*}
\end{lemma}

Новизна этой простой леммы состоит в том, что не обязательно брать шар в точке 0, суть остаётся такой же, но чуть-чуть ухудшается норма.

\begin{proof}
    \begin{gather*}
        \text{пусть } z \in X, \norm{z} < e, a \in B_\varepsilon(a), a + z \in B_\varepsilon(a) \\
        z = (a+z) - a \Rightarrow \norm{Uz} \leq \norm{U(a+z)} + \norm{U(a)} \leq R + R = 2R \\
        \text{пусть } x \in X, x \ne 0 \: \norm{x} < 1 \Rightarrow \norm{\varepsilon x} < \varepsilon \Rightarrow \norm{U(\varepsilon x)} \leq 2R \Rightarrow \norm{Ux} \leq \frac{2R}{\varepsilon} \\
        \norm{U} = \sup_{\seq{x \in X: \norm{x} \leq 1}} \norm{Ux} \leq \frac{2R}{\varepsilon}
    \end{gather*}
\end{proof}


\begin{proof}[Доказательство теоремы]
    Вспомним теорему Бэра о категориях (тык \ref{chap3:baire}): полное метрическое пространство нельзя представить как счётное объединение всюду плотных множеств.
    \begin{gather*}
        n \in \bN, D_n = \seq{ y \in Y: \norm{y} \leq n} \\
        \Rightarrow U^{-1}_\alpha(D_n) \text{ --- замкнутое множество в } X, \alpha \in A \\
        E_n = \bigcap_{\alpha \in A} U^{-1}_\alpha (D_n), E_n \text{ --- замкнутое}
        \intertext{Проверим, что $X = \bigcup^\infty_{n=1} E_n$}
        \text{пусть } x \in X \Rightarrow \sup_{\alpha \in A} \norm{U_\alpha x} < + \infty \Rightarrow \: \exists n \in \bN \\ 
        \sup_{\alpha \in A} \norm{U_\alpha x} < n \Rightarrow x \in U^{-1}_\alpha (D_n) \: \forall \alpha \in A \\
        \Rightarrow x \in E_n \\
        X = \bigcup^\infty_{n=1} E_n, X \text{ --- банахово}, E_n \text{ --- замкнутые } \Rightarrow
        \intertext{по теореме Бэра о категориях $\Rightarrow$} 
        \exists n_0 : \Int(E_{n_0}) \ne \bZero, \text{ то есть} \\
        \exists B_\varepsilon(a) \subset E_{n_0} = \bigcap_{\alpha \in A} U^{-1}_\alpha(D_{n_0}) \Rightarrow \\
        U_\alpha(B_\varepsilon(a))\subset D_{n_0} \text{ по лемме } \Rightarrow \norm{U \alpha} \leq \frac{2n_0}{\varepsilon} \: \forall \alpha \in A 
    \end{gather*}
\end{proof}

\begin{corollary}[Принцип фиксации особенности]
    $X$ --- банахово, $Y$ --- нормированное, $\seq{U_\alpha}_{\alpha \in A}, U_\alpha \in \B(X,Y)$
    \[ \text{пусть } \sup_{\alpha \in A} \norm{U_\alpha} = +\infty \Rightarrow \: \exists x_0 \in X : \sup_{\alpha \in A} \norm{U_\alpha(x_0)} = +\infty \]
    
\end{corollary}

Предлагается доказать следующее утверждение
\begin{statement}
    В условиях следствия $E = \seq{x \in X: \sup_{\alpha \in A} \norm{U_\alpha x} = +\infty}$. Доказать, что
    \begin{enumerate} 
        \item $E$ --- всюду плотно в $X$ 
        \item $X \setminus E$ --- множество первой категории
    \end{enumerate}
\end{statement}


\begin{definition}[сильный предел]
    \[(X, \norm{\cdot}), (Y, \norm{\cdot}), \seq{U_n}^\infty_{n=1}, U_n \in \Lin(X,Y) \]
    Если $\forall x \in X \: \liml_{n \to \infty} U_n x = Ux$, то 
    $U$ --- \textbf{поточечный} (или сильный) предел $\seq{U_n}$. Обозначение $U= \slim U_n$ (s = strong)
\end{definition}
Он хоть и сильный, но куда слабее сходимости по норме.

Отметим простые свойства: 

\begin{enumerate}
    \item $U_n \in \Lin(X,Y) \Rightarrow U \in \Lin(X,Y)$
    \item Если $U_n, U \in \B(X,Y), \liml_{n \to \infty} {U-U_n} = 0 \Rightarrow$ 
    \[ \liml_{n \to \infty} U_nx = Ux \: \forall x \in X \] 
\end{enumerate}

\begin{proof}[2]
    $\norm{Ux - U_n x} \leq \underbrace{\norm{U - U_n}}_{\to 0} \cdot \norm{x} \Rightarrow \liml_{n \to \infty} U_n x = Ux $
\end{proof}

\begin{remark}
    $U = \slim U_n \not \Rightarrow \liml_{n \to \infty} \norm{U - U_n} = 0$
\end{remark}

Пример, где поточечный предел существует и равен нулю, а предела по норме не существует
\begin{example}
    \begin{gather*}
        X = l^1 = \seq{x = \seq{x_n}^\infty_{n=1}, x_n \in \bC, \norm{x}_l = \sum^\infty_{n=1} \abs{x_n}} \\
        f_n: l^1 \rightarrow \bC \quad f_n \in (l^1)^* \quad f_n(x) = x_n, \norm{f_n} = 1 \\
        x \in l^1 \Rightarrow \liml_{n \to \infty} x_n = 0 \Rightarrow \liml_{n \to \infty} f_n(x) = 0 \: \forall x \in l^1 \\
        \bZero = \slim f_n, \text{ но } \norm{f_n - \bZero} = \underbrace{\norm{f_n}}_{\not \to 0} = 1
    \end{gather*}
\end{example}

\begin{example}
    $H$ --- сепарабельное гильбертово пространство, $\seq{e_n}^\infty_{n=1}$ --- ортонормированный базис.
    \begin{gather*}
        x \in H, x = \sum^\infty_{k=1} (x,e_k) e_k \quad S_n(x) = \sum^n_{k=1} (x,e_k) e_k \\
        \Rightarrow \forall x \in H \: \liml_{n \to \infty} S_n x = x, Ix = x \forall x \in H (I \text{ --- тождественный}) \\
        \Rightarrow I = \slim S_n \\
        (I-S_n)(e_{n+1}) = I(e_{n+1}) = e_{n+1} \Rightarrow \norm{I - S_n} = 1 \\
        \norm{I - S_n} \underset{n \to \infty}{\not \longrightarrow} 0
    \end{gather*}
\end{example}

Несмотря на то, что сильная сходимость слабее сходимости по норме, иногда оказывается, что сильный предел является непрерывным оператором.

\begin{theorem}
    $(X, \norm{\cdot})$ --- банахово, $(Y, \norm{\cdot})$ --- нормированное
    $U_n \in \B(X,Y), \text{ пусть } U = \slim U_n \Rightarrow$ 
    \[ U \in \B(X,Y), \norm{U} \leq \underline{\liml} \norm{U_n} \leq \sup_{n \in \bN} \norm{U_n} < + \infty \]
\end{theorem}

\begin{proof}
    Собираемся изо всех сил использовать принцип равномерной ограниченности. $\forall x \: \exists \liml_{n \to \infty} U_n(x) \Rightarrow \sup_{n} \norm{U_n x} < + \infty$.
    По принципу $\Rightarrow \sup_n \norm{U_n} < + \infty$
    \begin{gather*}
        \text{пусть } b = \underline{\liml} \norm{U_n} \Rightarrow \: \exists \seq{U_{n_k}} : b = \liml_{k \to \infty} \liml_{k \to \infty} \norm{U_{n_k}} \\
        \text{пусть } x \in X \Rightarrow Ux = \liml_{k \to \infty} U_{n_k}(x) \Rightarrow \\
        \norm{Ux} = \liml_{k \to \infty} \norm{U_{n_k} x} \leq \liml_{k \to \infty} \norm{U_{n_k}} \cdot \norm{x} = b \norm{x} \: \forall x \in X \\
        \Rightarrow U \in \B(X,Y) , \norm{U} \leq b
    \end{gather*} 
\end{proof}

\begin{remark}
    $U = \slim U_n$, возможно $\norm{U} < \underline{\lim} \norm{U_n}$
\end{remark}

\begin{example}
    $f_n: l^1 \rightarrow \bC, f_n(x) = x_n, \bZero = \slim f_n, \norm{f_n} = 1 \: \forall n. \norm{\bZero} = 0$
\end{example}

\begin{theorem}[Банах-Штейнгауз, критерий существования сильного предела]
    $X,Y$ --- банахово, $U_n \in \B(X,Y)$. Для того чтобы существовал $\slim U_n$, необходимо и достаточно 
    \begin{enumerate}
        \item $\exists M > 0: \norm{U_n} \leq M \: \forall n \in \bN$ 
        \item $\exists E \subset X, E$ полное, то есть $\overline{\calL(E)} = X$. $\seq{U_nx}$ --- фундаментальная для $\forall x \in E$ 
    \end{enumerate}
\end{theorem}

Существует множество вариантов этой теоремы, и все они по-своему полезные, поэтому у нас будет очень много замечаний потом

\begin{proof}
    $\Rightarrow$ пусть $U = \slim U_n$, мы уже доказали, что $\sup_n\norm {U_n} < + \infty$, а второе утверждение очевидно\\
    $\Leftarrow$ \\
    Пусть $x \in \calL(E)$, то есть $x = \sum^N_{k=1} c_k x_k, \: c_k \in \bC, x_k \in E$
    \begin{gather*}
        \norm{U_n(x) - U_m(x)} \leq \sum^N_{k=1} \abs{c_k} \cdot \norm{U_n x_k - U_m x_k} \Rightarrow \seq{U_n x} \text{ --- фундаментальная}
        \intertext{Пусть $x \in X$, проверим, что $\seq{U_nx}^\infty_{n=1}$ фундаментальна}
        x \in X, \varepsilon > 0 \quad \exists z \in \calL(E), \norm{x-z} < \varepsilon \\
        \exists N \in \bN \: n, m > N \Rightarrow \norm{U_n z - U_mz} < \varepsilon \\ 
        \norm{U_nx - U_mx} \leq \underbrace{\norm{U_nx - U_nz}}_{\leq \norm{U_n} \cdot \norm{x-z} \leq M \varepsilon} + \underbrace{\norm{U_nz - U_mz}}_{\varepsilon}
        + \underbrace{\norm{U_mz - U_mx}}_{\leq M\varepsilon} \\
        < \varepsilon(2M+1) \Rightarrow \seq{U_nx}^\infty_{n=1} \text{ фундаментальна} \\
        Y \text{ банахово} \Rightarrow \: \exists \liml_{n \to \infty} U_n x \: \forall x \in X \Rightarrow \: \exists U = \slim U_n
    \end{gather*}
\end{proof}

\begin{remark}
    \begin{enumerate}
        \item $\Rightarrow$ $X$ --- банахово, $Y$ --- нормированное 
        \item $\Leftarrow$ $Y$ --- банахово, $X$ --- нормированное
        \item В условии 2 теоремы моэно сформулировать 2':
            \[ \exists E \subset X, \overline{\calL(E)} = X, \: \exists \liml_{n \to \infty} U_n x \] 
    \end{enumerate}
\end{remark}

Заплатим жестокую цену за такую теорему: раньше $U$ не было, оно появлялось, критерий существования всё-таки, а здесь же мы предположим сразу непрерывность этого $U$
\begin{theorem}[Банах-Штейнгауз]
    $X$ --- банахово, $Y$ --- нормированное, $U_n \in \B(X,Y), U \in \B(X,Y)$ \\
    $ U = \slim U_n \Leftrightarrow$ 
    \begin{enumerate}
        \item $\sup_n \norm{U_n} \leq M < +\infty$ 
        \item $\exists E \subset X, \overline{\calL(E)} = X, \: \forall x \in E \: \exists \liml_{n \to \infty} U_n x $
    \end{enumerate}
\end{theorem}

\begin{proof}
    $\Rightarrow$ очевидно \\ 
    $\Leftarrow$  \\
    \begin{gather*}
        \forall x \in \calL(E) \Rightarrow \: \exists \liml_{n \to \infty} U_nx \\
        \text{пусть } x \in X, \varepsilon > 0 \: \exists z \in \calL(E), \norm{x-z} < \varepsilon, \: \exists N: n \geq N \Rightarrow \norm{Uz-U_nz} < \varepsilon \\
        \norm{Ux - U_nx} \leq \underbrace{\norm{Ux-Uz}}_{\leq \norm{U} \cdot \norm{x-z} \leq \norm{U}\varepsilon}+\underbrace{\norm{Uz-U_nz}}_{<\varepsilon} + \underbrace{\norm{U_nz-U_nx}}_{\norm{U_n} \cdot \norm{x-z} \leq M \varepsilon} 
        \intertext{мы тут сразу пользуемся тем, что $U$ --- непрерывный оператор и у него есть норма} 
        \leq \varepsilon(1 + \norm{U} + M) \Rightarrow \: \exists \liml_{n \to \infty} U_n x = Ux \: \forall x \in X
    \end{gather*}
\end{proof}

Маленькая историческая байка из серии <<Мифы и легенды из жизни Банаха>> о встрече Банаха и Штейнгауза .
Банах чудесным образом родился, никто не знает его мать, имя ему досталось от отца Стефана его крестили и оставили (Банах --- это фамилия женщины, которая заботилась о нём с трехдневного возраста). В школе Банах интересовался только матемтикой,
но рядом не оказалось никого, кто сказал бы ему идти на математический факультет, а ведь в Варшаве был хороший университет, преподавал там ученик Гаусса. В итоге Банах закончил что-то вроде политеха, издали 
интересовавшись математикой. И вот, началась первая мировая война, Банаха не взяли в армию, а Штейнгауза взяли, но потом отослали обратно.\\


Штейнгауз как-то шёл по улице и услышал, как 2 человека на ке что-то обсуждают, а доносятся от них умные типа <<Мера Лебега>>, Штейнгауз
обратился к ним: <<Предмет вашей учебной беседы настолько интересен!>>. И начал он им навешивать всякую математическую лапшу на уши. Через несколько дней Банах решил какую-то задачку, и Штейнгауз у себя на дому устраивает встречу математиков, 
они даже собирались организовать вчетвером краковское математическое ощество, но сам он потом уехал во Львов, перетащил туда Банаха. Банах устроился работать в какой-то из университетов и стал ликвидировать свою математическую безграмотность.
Штейнгауз же потом говорил, что главный его вклад в функциональный анализ --- это открытие Банаха.


\section{Применение принципа равномерной ограниченности к рядам Фурье}

Пусть $f \in \tilde{C}[-\pi, \pi]$. Волна означает $f(-\pi) = f(\pi)$
\begin{gather*}
    C_n(f) = \frac{1}{2\pi} \int^{\pi}_{-\pi} f(x)e^{-inx}dx \\ 
    S_n(f,x) = \sum^n_{k=-n} c_k e^{-ikx} = \int^\pi_{-\pi} f(t) D_n(x-t) dt \quad c_k = \frac{1}{2\pi} \int^\pi_{-\pi} f(t) e^{ikt} dt \\
    \text{ где } D_n(u) = \frac{1}{2\pi} \sum^n_{k=-n} e^{iku} = \frac{1}{2\pi} \frac{\sin(\left(n+\frac{1}{2}\right)u)}{\sin \left( \frac{u}{2} \right)} \text{ --- ядро Дирихле} \\
    S_n(e^{ikx}) = e^{ikx} \text{ если } n \geq k \Rightarrow \liml_{n \to \infty} S_n (e^{ikx}) = e^{ikx} \\
    If = f, I \text{ --- тождественный в } \tilde{C}[-\pi,\pi] \\
    \liml_{n \to \infty} S_n(e^{ikx}) = I(e^{ikx}) = e^{ikx} \Rightarrow \\
    \seq{e^{ikx}}_{k \in \bZ} \text{ --- полная система в } \tilde{C}[-\pi,\pi] \text{ по теореме Вейертршрасса} \\
    I(f) = \liml_{n \to \infty} S_n(f), \text{ при } f = e^{ikx} \text{ или } f \in \calL{e^{ikx}}_{k \in \bZ}
\end{gather*}

\begin{theorem}
    \begin{enumerate}
        \item $\exists f \in \tilde{C}[-\pi,\pi]$ т.ч. $S_n(f,x)$ не сходятся равномерно, более того $\sup_n \norm{S_n(f,x)}_\infty = +\infty$ (Лебег, 1906)
        \item пусть $x_0 \in [-\pi,\pi] \Rightarrow \: \exists f \in \tilde{C}[-\pi,\pi]$ т.ч. не $\exists \liml_{n \to \infty} S_n(f,x_0)$, более того $\sup_n \abs{S_n(f,x_0)} = +\infty$  (Дю Буа Реймонд, 1886)
    \end{enumerate}
\end{theorem}

Для доказательства будем применять следствия из принципа равномерной ограниченности. 
Если $S_n$ вычислять на базисе $e^{ikx}$, то сходимость будет, но теорему Банаха-Штейнгауза нельзя применять, потому что нет ограниченности по норме оператора $S_n$

\begin{proof}[Доказательство теоремы Лебега]
    Проверим, что $\sup_n \norm{S_n}_{\B(\tilde{C}[-\pi,\pi])} = \infty$
    \begin{gather*}
        S_n(f,x) = \int^\pi_{-\pi} = f(t) D_n(x-t)dt \quad S_n \in \Lin(\tilde{C}[-\pi,\pi]) \\
        f \in \tilde{C}[-\pi,\pi], x \in [-\pi,\pi]
        \intertext{мы уже вычисляли норму интегрального оператора в пространстве непрерывных функций}
        \norm{S_n} = \max_{x \in [-\pi,\pi]} \int^\pi_{-\pi} \abs{D_n(x-t)}dt
        \intertext{цель ближайших вычислений: проверить, что при $n \to \infty$ норма стремится к $\infty$. Сначала проверим, что она не зависит от $x$}
        \int^\pi_{-\pi} \abs{D_n(x-t)}dt = [[\tau = x - t, d\tau = -dt]] = - \int^{x-\pi}_{x+\pi} \abs{D_n(\tau)}d\tau = [[D_n \text{ -- } 2\pi\text{-периодическая}]] \\
         \int^\pi_{-\pi} \abs{D_n(\tau)}d\tau = [[\text{чётная}]] \\
        = \frac{1}{\pi} \int^\pi_{0} \frac{\sin(n+\frac{1}{2})\tau}{\sin(\frac{\tau}{2})} d\tau \geq \frac{2}{\pi} \int^\pi_0 \frac{\abs{\sin(n+\frac{1}{2})}}{\tau}d\tau = \\
        [[x \geq \sin x \quad v = \left( n + \frac{1}{2} \right) \Rightarrow \frac{d\tau}{\tau} = \frac{dv}{v}]] = \\
        = \frac{2}{\pi} \int^{\left(n+\frac{1}{2}\right)\pi}_0 \frac{\abs{\sin v}}{v} dv \geq \frac{2}{\pi} \sum^n_{k=1} \int^{k\pi}_{(k-1)\pi} \frac{\abs{\sin v}}{v} dv \geq \\
        [[v \in [(k-1)\pi,k\pi] \Rightarrow v \leq k\pi \Rightarrow \frac{1}{v} \geq \frac{1}{k\pi} ]] \\
        \geq \frac{2}{\pi} \cdot \frac{1}{\pi} \underbrace{\sum^n_{k=1} \frac{1}{k}}_{\sigma_n} \geq \frac{2}{\pi^2} \sigma_n, \liml_{n \to \infty} \sigma_n = +\infty \\
        x \in [k, k+1] \Rightarrow k \leq x \Rightarrow \frac{1}{k} \geq \frac{1}{x} \Rightarrow \frac{1}{k} \geq \int^{k+1}_k \frac{dx}{x} \Rightarrow \\
        \sigma_n = 1 + \ldots + \frac{1}{n} \geq \int^{n+1}_1 \frac{dx}{x} = \ln(n+1) \\
        \Rightarrow \norm{S_n} \geq \frac{2}{\pi^2}
    \end{gather*}
    как используем принцип фиксации особенности? Есть последовательность операторов $S_n$. При фиксированном $f$ она стремится к  $S_n(f)$. Мы оценили снизу норму и доказали, что она стремится к бесконечности. Если норма операторов не ограничена, 
    то в нашем банаховом пространстве непрервны $2\pi$-периодических функций найдется такой элемент, на котором норма не ограничена, значит там тем более не может быть равномерной сходиомсти.
\end{proof}

\begin{proof}[Доказательство теоремы Дю Буа Реймонда]
    Вместо линейных операторов теперь рассмотрим линейные функционалы. $x_0$ --- фиксирована, $S_n(f,x_0)$ --- линейные функционалы. $S_n(f,x_0) : \tilde{C}[-\pi,\pi] \rightarrow \bC$.
    Там же, где мы вычисляли норму интегрального оператора, мы вычисляли норму линейного функционала (ссылку)
    \begin{gather*}
        S_n(f,x_0) = \int^\pi_{-\pi} f(t) D_n(x_0-t) dt \\
        \text{норма функционала } \norm{S_n}_{(\tilde{C}[-\pi,\pi])^*} = \int^\pi_{-\pi} \abs{D_n(x_0-t)} dt \stackrel{\text{Лебег}}{=} \\
        = \norm{S_n} \geq \frac{2}{\pi^2} \ln n \Rightarrow \sup_n \norm{S_n(f,x_0)}_{(\tilde{C}[-\pi,\pi])^*} = +\infty
        \intertext{по принципу фиксации особенности} 
        \Rightarrow \: \exists f \in \tilde{C}[-\pi,\pi] : \sup_n \abs{S_n(f,x_0)} = +\infty
    \end{gather*}
\end{proof}

\begin{remark}
    Пусть $E \subset [-\pi,\pi], E$ --- счётное $\Rightarrow \: \exists f \in \tilde{C}[-\pi,\pi] \: \forall x_0 \in E \: \sup_n \abs{S_n(f,x_0)} = +\infty$
\end{remark}

\begin{remark}
    $\exists f \in L^1[-\pi,\pi] \: \forall x \in [-\pi, \pi]$ не $\exists \liml_{n \to \infty} S_n(f,x)$
\end{remark}

Этот пример построил Колмогоров  в 1926 году, а в 1923 построил такую функцию, у которой почти всюду не $\exists \liml$. Колмогоров был учеником Лузина. А он предъявил такую гипотезу

\begin{remark}[Гипотеза Лузина, 1923]
    $f \in L^2[-\pi,\pi] \Rightarrow S_n(f,x) \rightarrow f(x)$ почти всюду на $[-\pi,\pi]$.
\end{remark}


Шведский математик Карлесон В 22 года подумал улучшить пример Колмогорова. Поехал в Америку на семинар по тригонометрическим рядам, там рассказал Зигмунду, как собирается опровергать гипотизу Лузина. А тот его всячески поощрял. Весь мир тогда считал, что гипотеза неверна.

Вот он несколько лет мучился и подумал, что  функция, которую он пытается построить, не существует. Так и оказалось.
На международном математическом конгрессе в 1966 Карлесон показал, что гипотеза верна. Колмогоров вышел, пожал ему руку и сказал, что это главный
результат математического анализа за весь XX век.

\begin{lemma}[Риман-Лебег]
    $ f \in L^1[-\pi,\pi]$
    \[c_n(f) = \frac{1}{2\pi} \int^\pi_{-\pi} f(x)e^{-inx} dx \Rightarrow \liml_{n \to \infty} c_n(f) = 0 \]
\end{lemma}


\begin{proof}
    Будем думать, что $c_n \in \Lin(L^1, \bC)$  --- линейный функционал. При фиксированном $f c_n$ --- конкретное число
    \begin{gather*}
        f \in L^1 \quad \abs{c_n(f)} \leq \frac{1}{2\pi} \int^\pi_{-\pi} \abs{f(x)} dx = \frac{1}{2\pi} \norm{f}_1 \\
        \Rightarrow c_n \in (L^1)^*, \norm{c_n}_{(L^1)^*} \leq \frac{1}{2\pi} \\
        \seq{ \chi_{[a,b)}} \text{ --- полное семейство в } L^1[-\pi,\pi], -\pi \leq a < b \leq \pi \\
        c_n(\chi_{[a,b)}) = \frac{1}{2\pi} \int^\pi_{-\pi} \chi_{[a,b)}(x) e^{-inx}dx = \frac{1}{2\pi} \int^b_a e^{-inx} dx = \\
        = \frac{1}{2\pi} \frac{e^{-inb} - e^{-ina}}{-in} \Rightarrow \\ 
        \abs{c_n(\chi_{[a,b)})} \leq \frac{2}{2\pi n} = \frac{1}{\pi n} \underset{n \to \infty}{\longrightarrow} 0 \\
        c_n(\chi_{[a,b)}) \longrightarrow \bZero(\chi_{[a,b)}) \Rightarrow \\
        \text{ по теореме Банаха-Штейнгауза } \forall f \in L^1 \quad c_n(f) \underset{n \to \infty}{\longrightarrow} 0
    \end{gather*}
\end{proof}


\end{document}